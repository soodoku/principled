Partisans dislike supporters of the opposing party \citep{iyengar2012,
iyengar2013}. Sizable proportions of Democrats and Republicans say they would be
unhappy if a family member of theirs married someone from the opposing party
\citep{iyengar2012}. This widespread mutual antipathy exists despite a vast overlap in policy positions of supporters of both parties \citep{fiorina2012}. If policy differences don't explain this affective gulf, what does?

Ideologically extreme partisan elites draw considerable support from their own
party \citep{sood2013}. However, the support is not as well-founded in actual
agreement over policy as it is in \textit{perceived} agreement over policy. As
we later show, partisans also tend not to penalize elites of their own party for
their missteps. Whatever the reasons for tepid reactions to prominent missteps
by policitians of their own party, be it ignorance or partisan
selective exposure or greater faith in denials from one's own party
members, we show that information about these tepid reactions riles the
supporters of the opposing party.

We start by documenting partisans' tepid reactions to missteps by their own
party's elites. We document this using a new database of political scandals, coupled with opinion data from 2000 state level polls measuring approval of Senators, Governors, and the President. We supplement the public opinion data with national panel data from the National Annenberg Election Studies (NAES) and the National Election Studies. We use an interrupted time-series design to measure public opinion before and after political scandals. We find that partisans tend not to update evaluations of embattled leaders. Next, we experimentally manipulate the level of approval that leaders enjoy among their party's supporters following political debacles. We find that negative evaluations of outparty supporters increase when they continue to support embattled leaders, vis-a-vis when outparty supporters withdraw their support.

\section*{Partisans Respond to Partisan Response}
We randomly assign respondents to read about differing partisan responses to party elites' political errors. We thus create a counterfactual for identifying the effect of perceived outparty intransigence on affect toward the outparty.

\subsection*{Research Design}
In November 2013, we recruited 930 survey participants through Amazon's
Mechanical Turk \citep[see][]{BerinskyHuberLenz2012}. We randomly assigned participants to read a news story on one of three ongoing political scandals at the time: the troubled rollout of the U.S. health exchange website (Democratic Party blunder), Ted Cruz's inflammatory rhetoric precluding compromise during the shutdown and debt ceiling negotiations (Republican Party blunder), and Toronto mayor Rob Ford's drug scandal (control condition). Within news stories on American party leaders' political errors, we randomly assigned participants to one of two conditions related to partisan reaction. In one condition (``response''), participants read that support for the embattled elite (Obama or Cruz) declined among co-partisan citizens. In the other condition, (``no response''), participants read that support for the elite remained steadily high. See Appendix A for a complete transcript of the stories.

We assessed whether exposure to stories in which partisans withdraw support from
their party's elites when they learn about the political scandal, compared to
both, the control group, and the condition in which partisans continue to
support their leaders despite the political scandal, changed what partisans of
the opposing party believed of the supporters of the party whose elite was
embroiled in the scandal. In particular, after exposing respondents to the
stories, we asked the respondents how well they thought each of the following
eight traits -ignorant, sincere, open to reason, smug, selfish, patriotic,
compassionate, and hypocritical- described ``people who support the Democratic
(Republican) Party'' on a semantic five-point scale, ranging from ``extremely
well'' to ``not at all.'' We recoded ratings on negatively-valenced traits so
that all trait ratings ranged from negative to positive. We next created
a difference score as a summary measure of partisan affect (The $\alpha$ for
difference ratings was .90).

On the thinking that the effects would be moderated by respondent's level of
political interest, and political knowledge, we also measured these variables. 
Respondent's partisan identification was measured using the conventional
branched question found in the NES.
