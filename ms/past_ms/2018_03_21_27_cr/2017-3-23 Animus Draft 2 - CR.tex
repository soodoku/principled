\documentclass[12pt]{article}
\usepackage{times}
\usepackage[margin=1in]{geometry}
\usepackage[linkcolor=blue,
colorlinks=true,
urlcolor=blue,
citecolor=blue,
pdftitle={Xperceive}]{hyperref}

%\usepackage[nolists]{endfloat}
\usepackage{indentfirst}
\usepackage{booktabs}
\usepackage{setspace} 
\usepackage{verbatim}
\usepackage{fancyvrb}
\usepackage{longtable}
\usepackage{lscape}



\usepackage[none]{hyphenat}
\usepackage{titlesec}
\titleformat{\section}{\normalfont \centering}{\thesection}{.5em}{}
\titleformat{\subsection}{\normalfont}{\thesection}{.5em}{}
\setlength{\footnotesep}{0.5cm}

\let\proglang=\texttt
\newcommand{\pkg}[1]{{\fontseries{b}\selectfont #1}}

\newcommand{\superscript}[1]{\ensuremath{^{\textrm{#1}}}}

\usepackage{graphicx}
\usepackage{rotating}

\usepackage{bm}
\usepackage{amsfonts,amssymb,amsbsy}
\usepackage{amsxtra}

\usepackage{dcolumn}
\newcolumntype{.}{D{.}{.}{5}}
\newcolumntype{q}{D{.}{.}{3}}
	
\usepackage{natbib}
\bibpunct[, ]{(}{)}{;}{a}{}{,}

\usepackage{amsmath}

\raggedright
\parindent=1.5em % <- or whatever indent you want


\begin{document}
\doublespacing
\vspace{2cm}

\begin{center}
Explaining Partisan Affect\\
\vspace{2cm}
\today\\\vspace{2cm}

Doug Ahler\footnote{Doug Ahler is a Ph.D. candidate in Political Science at the
University of California, Berkeley. Doug can be reached at
\href{mailto:dahler@berkeley.edu}{dahler@berkeley.edu}} and Gaurav
Sood\footnote{Gaurav Sood is a National Fellow at the Hoover Institution at
Stanford University. Gaurav can be reached at
\href{mailto:gsood@stanford.edu}{gsood@stanford.edu}}

\end{center}

\setcounter{page}{0}
\thispagestyle{empty}
\thispagestyle{empty}

\newpage
\setcounter{footnote}{1}


\begin{comment}
	sweaver(paste0(basedir, "xperceive/partisanResponse"), "presponse")
\end{comment}
\newpage

\doublespacing
Despite the absence of widespread mass-level polarization on political issues \citep{fiorina2012}, partisan identifiers increasingly dislike supporters of the opposing side \citep{iyengar2012, iyengar2013}. Partisans report not only greater disdain for members of the outparty, but also a greater sense of social distance. Tellingly, significant portions of Democrats and Republicans cringe at the idea of a family member marrying an outparty supporter \citep{iyengar2012}. This rising partisan antipathy promises a slew of unappealing consequences for the future of political civility and democratic deliberation. 

If massive political differences fail to explain this increasing affective gulf, what does? One potential source of disaffection between partisans concerns not attitudes and opinions themselves but rather the basis on which they are held. Partisans notoriously filter information through a ``perceptual screen,'' yielding attitudes and beliefs that prioritize consistency with partisan identity over accuracy and cool reason \citep{TAV,TAVR,Bartels2002}. While the perceptual screen operates largely subconsciously (c.f. \citet{BullockEtAl2013}), Democrats observe and lament Republicans' closed-mindedness and vice versa \citep{Haidt2011}. (Say something here about our own data from the control condition???)

To determine whether perceptions of the outparty as biased and unresponsive to political facts produce enmity between Democrats and Republicans, we conducted two studies. In the first, we rely on an interrupted time-series design to measure public opinion before and after political scandals. We find that partisans tend not to update evaluations of embattled leaders, and that supporters of the other side tend to evaluate that party's supporters more negatively. In the second study, we experimentally manipulate the level of approval that leaders enjoy among their party's supporters following political debacles. We find that negative evaluations of outparty supporters increase when they continue to support embattled leaders, but decrease when they are responsive to political facts. 

\section*{Study 1}

\section*{Study 2: Survey Experimental Evidence}
Study 1 demonstrates that partisans' feelings toward their outparty peers decline when scandals and political snafus plague outparty leaders. (We hope.) While this evidence confirms that events in the elite sphere shape mass-level partisan affect, we have yet to pin down our hypothesized mechanism. To do so, we rely on a survey experiment in which we randomly assign respondents to read about differing partisan responses to party elites' political errors. We thus create a counterfactual for identifying the effect of perceived outparty intransigence on affect toward the outparty.

\subsection*{Research Design}
In November 2013, we recruited XXX survey participants through Amazon's Mechanical Turk. We randomly assigned participants to read a news story on one of three ongoing political snafus at the time: the troubled rollout of the U.S. health exchange website (Democratic Party blunder), Ted Cruz's inflammatory rhetoric precluding compromise during the shutdown and debt ceiling negotiations (Republican Party blunder), and Toronto mayor Rob Ford's drug scandal (control condition). Within news stories on American party leaders' political errors, we randomly assigned participants to one of two conditions related to partisan reaction. In one condition (``response''), participants read that support for the embattled elite (Obama or Cruz) declined among co-partisan citizens. In the other condition, (``no response''), participants read that support for the elite remained steadily high. See the SI to read the stories in full.



\section*{Measures}

\textbf{Trait Ratings}: Respondents were asked about traits of those who
support different parties.

\textbf{Thermometer Ratings}: Respondents were asked how warm or cold they felt
towards the different parties on a scale of 0 (extremely cold) to 100 (extremely
warm).

\textbf{Education}: Respondent's level of education was measured in a variety
of ways but eventually collapsed into three categories: less than high
school, some college, or college and beyond.

\section*{Analysis}

\section*{Results}

Among Democrats, DV = rdtrt

Compared to control, R No Change causes increase in antipathy towards
Republicans (b = .06, p < .11).

Compared to R Change, R No Change causes increase in antipathy towards
Republicans (b = .07, p < .06).

Effect is substantially greater among Strong/Weak Democrats (taking out
leaners), and among those with at least some basic levels of political
knowledge.

\clearpage 
\section*{Tables}
\begin{table}[!ht]
\caption{Within Democrats}
\label{} 
\begin{tabular}{ l D{.}{.}{2}D{.}{.}{2}D{.}{.}{2} } 
\hline 
  & \multicolumn{ 1 }{ c }{ R-D Trt } & \multicolumn{ 1 }{ c }{ R-D Trt } & \multicolumn{ 1 }{ c }{ R-D Trt } \\ \hline
 %                & R-D Trt & R-D Trt & R-D Trt\\ 
(Intercept)      & 0.35 ^* & 0.35 ^* & 0.34 ^*\\ 
                 & (0.02)  & (0.03)  & (0.02) \\ 
treatD Change    & -0.06   &         &        \\ 
                 & (0.03)  &         &        \\ 
treatD No Change & -0.03   &         &        \\ 
                 & (0.03)  &         &        \\ 
treatR Change    & -0.01   & -0.01   &        \\ 
                 & (0.03)  & (0.03)  &        \\ 
treatR No Change & 0.06    & 0.06    & 0.07   \\ 
                 & (0.04)  & (0.04)  & (0.03)  \\
 $N$              & 552     & 319     & 219    \\ 
$R^2$            & 0.02    & 0.01    & 0.02   \\ 
adj. $R^2$       & 0.02    & 0.01    & 0.01   \\ 
Resid. sd        & 0.25    & 0.25    & 0.26    \\ \hline
 \multicolumn{4}{l}{\footnotesize{Standard errors in parentheses}}\\
\multicolumn{4}{l}{\footnotesize{$^*$ indicates significance at $p< 0.05 $}} 
\end{tabular} 
 \end{table}\begin{table}[!ht]
\caption{Within Democrats}
\label{} 
\begin{tabular}{ l D{.}{.}{2}D{.}{.}{2}D{.}{.}{2} } 
\hline 
  & \multicolumn{ 1 }{ c }{ Str./Weak Dems. } & \multicolumn{ 1 }{ c }{ Strong Dems. } & \multicolumn{ 1 }{ c }{ Know > 0 } \\ \hline
 %                & Str./Weak Dems. & Strong Dems.    & Know > 0       \\ 
(Intercept)      & 0.36 ^*         & 0.38 ^*         & 0.35 ^*        \\ 
                 & (0.03)          & (0.04)          & (0.04)         \\ 
treatR Change    & 0.01            & 0.07            & 0.02           \\ 
                 & (0.04)          & (0.05)          & (0.05)         \\ 
treatR No Change & 0.11 ^*         & 0.25 ^*         & 0.12 ^*        \\ 
                 & (0.04)          & (0.05)          & (0.05)          \\
 $N$              & 238             & 105             & 157            \\ 
$R^2$            & 0.03            & 0.18            & 0.04           \\ 
adj. $R^2$       & 0.03            & 0.16            & 0.03           \\ 
Resid. sd        & 0.26            & 0.23            & 0.25            \\ \hline
 \multicolumn{4}{l}{\footnotesize{Standard errors in parentheses}}\\
\multicolumn{4}{l}{\footnotesize{$^*$ indicates significance at $p< 0.05 $}} 
\end{tabular} 
 \end{table}\begin{table}[!ht]
\caption{Within Democrats}
\label{} 
\begin{tabular}{ l D{.}{.}{2}D{.}{.}{2}D{.}{.}{2} } 
\hline 
  & \multicolumn{ 1 }{ c }{ Rep Trt } & \multicolumn{ 1 }{ c }{ R-D Trt } & \multicolumn{ 1 }{ c }{ R-D Trt with Pid7 } \\ \hline
 %                     & Rep Trt           & R-D Trt           & R-D Trt with Pid7\\ 
(Intercept)           & 0.63 ^*           & 0.29 ^*           & 0.46 ^*          \\ 
                      & (0.02)            & (0.02)            & (0.06)           \\ 
treatR No Change      & 0.05 ^*           & 0.10 ^*           & 0.28 ^*          \\ 
                      & (0.02)            & (0.04)            & (0.09)           \\ 
pid7                  &                   &                   & -0.08 ^*         \\ 
                      &                   &                   & (0.03)           \\ 
treatR No Change:pid7 &                   &                   & -0.09 ^*         \\ 
                      &                   &                   & (0.04)            \\
 $N$                   & 219               & 219               & 219              \\ 
$R^2$                 & 0.02              & 0.04              & 0.18             \\ 
adj. $R^2$            & 0.01              & 0.03              & 0.17             \\ 
Resid. sd             & 0.18              & 0.26              & 0.24              \\ \hline
 \multicolumn{4}{l}{\footnotesize{Standard errors in parentheses}}\\
\multicolumn{4}{l}{\footnotesize{$^*$ indicates significance at $p< 0.05 $}} 
\end{tabular} 
 \end{table}\clearpage
\singlespacing

\begin{tabular}{lcccc} \hline
 & (1) & (2) & (3) & (4) \\
 & basic1 & basic2 & basic3 & basic4 \\
VARIABLES & ft\_dem & ft\_rep & ft\_dem & ft\_rep \\ \hline
 &  &  &  &  \\
reps\_ignore & -2.635 & 1.222 & -4.191 & 1.375 \\
 & (2.918) & (2.970) & (5.314) & (4.857) \\
reps\_react & -6.026** & 0.428 & 2.942 & -3.337 \\
 & (2.749) & (2.799) & (5.190) & (4.744) \\
dems\_ignore & -1.102 & 4.075 & -0.972 & -2.880 \\
 & (2.770) & (2.820) & (5.409) & (4.943) \\
dems\_react & -5.317* & 2.920 & 6.640 & 8.991* \\
 & (2.798) & (2.848) & (5.409) & (4.943) \\
Constant & 68.010*** & 22.580*** & 30.585*** & 63.171*** \\
 & (2.042) & (2.078) & (3.549) & (3.244) \\
 &  &  &  &  \\
Observations & 552 & 552 & 172 & 172 \\
 R-squared & 0.013 & 0.006 & 0.025 & 0.042 \\ \hline
\multicolumn{5}{c}{ Standard errors in parentheses} \\
\multicolumn{5}{c}{ *** p$<$0.01, ** p$<$0.05, * p$<$0.1} \\
\end{tabular}

\renewcommand{\bibname}{References}
\bibliographystyle{apsr}
\bibliography{xperceive}

\end{document}
